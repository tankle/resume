\documentclass[10pt,a4paper,roman]{moderncv}

% moderncv themes
% classic casual
\moderncvstyle{classic}
% orange,green,red,purple,grey,black
\moderncvcolor{black}
%\renewcommand{\familydefault}{\sfdefault}
%\nopagenumbers{}

% character encoding
%\usepackage[utf8]{inputenc}
%\usepackage{CJKutf8}

% adjust the page margins
\usepackage[scale=0.85]{geometry}
\setlength{\hintscolumnwidth}{0.12\textwidth}
%\setlength{\makecvtitlenamewidth}{10cm}

\usepackage{fontspec}
\usepackage{xunicode}
\usepackage{xeCJK}
\setmainfont{Minion Pro}
\setsansfont{Myriad Pro}
\setCJKmainfont{Adobe Kaiti Std}
%\setCJKmainfont{STKaiti}

\usepackage{lastpage}
\usepackage{fancyhdr}
\pagestyle{fancy}
\fancyhf{}
\fancyfoot[L]{\footnotesize\textit{谭聪个人简历}}
%\fancyfoot[C]{\footnotesize\thepage/\pageref{LastPage}}
\fancyfoot[R]{\footnotesize\textit{Last updated: \today}}

\usepackage[UKenglish]{datetime}
\newdateformat{UKvardate}{%
\THEDAY\ \monthname[\THEMONTH] \THEYEAR}
\UKvardate

\usepackage{manfnt}

\newcommand{\hello}{{\tiny\textdbend}}

% personal data
\name{谭}{聪}
%\title{应届毕业生}
\address{深圳市南山区西丽大学城}{哈尔滨工业大学深圳研究生院计算机学院C栋}
\phone[mobile]{+86~159~9964~5033}
%\phone[fixed]{+86~027~6875~5072} % optional
%\phone[fax]{+86~027~6875~4150} % optional

\email{viptancong@163.com}
\homepage{tankle.github.io}
\social[github]{tankle}   

\extrainfo{1990-12 \\ 男}

%\photo[64pt][0.4pt]{tan.jpg}
%\quote{{\footnotesize Last updated: \today}}
\quote{{ 求职意向:自然语言处理工程师}}

% to show numerical labels in the bibliography (default is to show no labels); only useful if you make citations in your resume
% 显示论文编号
\makeatletter
\renewcommand*{\bibliographyitemlabel}{\@biblabel{\arabic{enumiv}}}
\makeatother

% bibliography with mutiple entries
%\usepackage{multibib}
%\newcites{book,misc}{{Books},{Others}}
%----------------------------------------------------------------------------------
%            content
%----------------------------------------------------------------------------------

\AfterPreamble{\hypersetup{
    pdfstartview={XYZ null null 1.00},
    baseurl={}
}}

\begin{document}
%-----       resume       ---------------------------------------------------------
\makecvtitle


\section{教育背景}
\cventry{2013--至今}{硕士 \textnormal{(计算机科学与技术)}}{哈尔滨工业大学深圳研究生院}{GPA: 3.516 专业前15\%(15/118)}{}{导师: 王晓龙教授 \newline 研究方向:问答系统,信息检索,机器学习,自然语言处理 \newline 毕业时间:2016年01月}
\cventry{2009--2013}{学士 \textnormal{(计算机科学与技术)}}{杭州电子科技大学}{成绩:专业前15\%}{}{}


\section{研究项目}
\cvitem{NTCIR-11 Temporalia}{时间信息检索评测:在搜索引擎的搜索过程中,考虑查询的时间属性和结果文档中的时间信息,使得搜索结果既内容相关又符合时间要求。我们的评测结果获得国际第二, 并发表论文一篇\cite{hou2014hitsz}。
\newline	主要工作:使用Solr对文档建立索引和检索,抽取问句和搜索结果的特征,包括文本特征和时间特征,使用Learning to Rank算法对搜索结果进行重排序。
\newline 	工作亮点:1. 将查询按照时间属性分类(过去、最近、将来、与时间无关);
\newline  2. 引入时间因素, 使答案既内容相关又时间相关。 }
\cvitem{SemEval-2015 task3}{社区问答中答案选择任务(Answer Selection in Community Question Answering),是指在社区问答中,对每个问题的答案质量进行判定。该任务包含两种语言:Arabic和English。在Arabic子任务中,我们组(HITSZ-ICRC)取得了第三名,在English子任务中,获得了第二名 \cite{hou-EtAl:2015:SemEval1}。
\newline	主要工作:抽取文本特征,比如BOW、tf-idf、共现词等,也尝试过结合word2vec词向量特征,设计分类算法进行分类。
\newline 	难点和亮点:1. 样本不平衡问题(Good,Protential, Bad,Dialogue)),设计了多种样本平衡算法;
\newline 2. 设计了一个树形层次分类方法,结合随机森林(Random Forest)、支持向量机(SVM)等算法。}
\cvitem{SemEval-2015 task5}{QA TempEval 是指以问答的方式来评价对文本中时间信息的理解程度。主要工作:1. 文本中时间的抽取和归一化,2. 事件的抽取, 3. 事件和时间、事件与事件之间关系的抽取\cite{hou-EtAl:2015:SemEval2}。
	\newline 主要难点:1. 时间短语的识别; 2. 关系的抽取;亮点:采取bagging方式来合并多个标注结果。}

\cvitem{开源Github}{1. 基于隐马尔科夫模型的中文分词工具(HMMSegment),python实现;
\newline 2. 实现常见的机器学习算法,如SVM,Adaboost,Kmeans,LDA等;
\newline 3. 为Stanford CoreNLP,timeml-qa,caevo等项目提交过代码,主要是修复一些bug等。}

%\section{已发表论文}
%, Yongshuai Hou, Xiaolong Wang, Qingcai Chen, Man Li, Cong Tan - , 2014.Machine Learning and Cybernetics 481, 
%\cvitem{\hello}{\href{http://link.springer.com/chapter/10.1007\%2F978-3-662-45652-1_6}{User Input Classification for Chinese Question Answering System}, Yongshuai Hou, Xiaolong Wang, Qingcai Chen, Man Li,\textbf{ Cong Tan }, \textit{Machine Learning and Cybernetics}, 481 (2014), 52-59}
%HITSZ-ICRC at NTCIR-11 Temporalia Task, Yongshuai Hou, Cong Tan, Jun Xu, Youcheng Pan, Qingcai Chen, Xiaolong Wang - Proceedings of the 11th NTCIR Conference, Tokyo, 2014.
%\cvitem{\hello}{\href{http://research.nii.ac.jp/ntcir/workshop/OnlineProceedings11/pdf/NTCIR/Temporalia/07-NTCIR11-TEMPORALIA-HouY.pdf}{HITSZ-ICRC at NTCIR-11 Temporalia Task}, Yongshuai Hou, \textbf{Cong Tan}, Jun Xu, Youcheng Pan, Qingcai Chen, Xiaolong Wang, \textit{Proceedings of the 11th NTCIR Conference, Tokyo}, 2014, 468-473}




\section{专业技能 }
\cvitem{\hello}{掌握C++、Java、Python  等编程语言;}
\cvitem{\hello}{掌握主要的数据结构和算法。本科期间获得浙江省第八届程序设计竞赛铜牌。}
\cvitem{\hello}{掌握常用机器学习算法、自然语言处理等相关技术;掌握多种分类算法,如SVM,Random Forest;熟练使用机器学习工具:libsvm,scikit-learn,weka等;了解深度学习(Deep Learning)。}
\cvitem{\hello}{熟悉Linux环境,了解常见的命令;熟悉git等版本管理工具}


\section{所获奖励和证书 }
\cvitem{2014-10}{哈尔滨工业大学深圳研究生院一等奖学金}
\cvitem{2014-10}{哈尔滨工业大学深圳研究生院三好学生}
\cvitem{2013-09}{哈尔滨工业大学深圳研究生院特等奖学金}
%\cvitem{2010-12}{英语四级}

\section{校园实践}
\cvitem{2014.12.5-9}{NLPCC 2014 和 ADL 52期志愿者。}
\cvitem{2014.01-2014.12}{深圳大学城微软技术俱乐部技术部部长,在担任技术部部长期间,组织了多次俱乐部活动,还获邀参加了2014年微软夏令营。}
\cvitem{2013.09-2014.09}{哈工大深圳研究生院研究生会学术部成员,组织过多项校内活动,比如辩论赛、科技展、主持人大赛等,并主持了主持人大赛。}


% Publications from a BibTeX file without multibib
%  for numerical labels: \renewcommand{\bibliographyitemlabel}{\@biblabel{\arabic{enumiv}}}% CONSIDER MERGING WITH PREAMBLE PART
%  to redefine the heading string ("Publications"): \renewcommand{\refname}{Articles}
% plain  ieeetr apalike abbrv alpha 

\renewcommand{\refname}{论文}
\nocite{*}
\bibliographystyle{plain}
\bibliography{publications}   


% 'publications' is the name of a BibTeX file

%\section{References}
%\cvitem{}{Available upon request}
%\cvitem{Reference~1}{Professor~Yeneng~Sun, Raffles Professor of Social Sciences, National University of Singapore \newline Department of Economics, National University of Singapore, 117570, Singapore \newline Phone: +65~6516~3941; E-mail: ynsun@nus.edu.sg}
%\cvitem{Reference~2}{Professor~Xiao~Luo, Associate~Professor, National University of Singapore \newline Department of Economics, National University of Singapore, 117570, Singapore \newline Phone: +65~6516~6231; E-mail: ecslx@nus.edu.sg}
%\cvitem{Reference~3}{Professor~Nicholas~C.~Yannelis, Henry B. Tippie Research Professor of Economics, University of Iowa \newline Department of Economics, Henry B. Tippie College of Business, \newline University of Iowa, Iowa City, IA 52242, United States \newline Phone: +1~319~335~2290; E-mail: nicholasyannelis@gmail.com}


\end{document} 